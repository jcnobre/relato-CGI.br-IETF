\documentclass[12pt]{article}

\usepackage{sbc-template}
\usepackage[utf8]{inputenc}
\usepackage{url}
%\usepackage[page]{appendix}
\usepackage{graphicx}

\title{Relato sobre a Participação no the Internet Engineering Task Force dos Projetos Aprovados na Chamada Pública 0001/2014 do Comitê Gestor da Internet no Brasil}

\author{J{\'e}ferson Campos Nobre\inst{1}, Lisandro Zambenedetti Granville\inst{1},\\ Klaus Wehmuth\inst{2}, Artur Ziviani\inst{2},\\ Marcelo Santos\inst{3}, Felipe Lopes\inst{3}, Leonidas Lima\inst{3}, Stenio Fernandes\inst{3}}

\address{Institute of Informatics -- Federal University of Rio Grande do Sul
  (UFRGS)
\nextinstitute
 Laboratório Nacional de Computação Científica (LNCC/MCTI)
\nextinstitute
 Universidade Federal de Pernambuco (UFPE)
 % TODO
  \email{\{jcnobre,granville\}@inf.ufrgs.br, }
}

\begin{document} 

\maketitle

\begin{abstract}

The present work is an experience report about the participation of the authors on the Internet Engineering Task Force (IETF). Such participation is funded by the Public Call 0001/2014 of the \textit{Comitê Gestor da Internet no Brasil} (CGI.br).

\end{abstract}

\begin{resumo}

O presente trabalho é um relato de experiência sobre a participação dos autores no \textit{Internet Engineering Task Force} (IETF). Tal participação é financiada pela Chamada Pública 0001/2014 do Comitê Gestor da Internet no Brasil (CGI.br).

\end{resumo}

\section{Introdução}

% http://cgi.br/noticia/cgi-br-divulga-resultado-da-selecao-publica-para-o-ietf-e-irtf/10043

O \textit{Internet Engineering Task Force} (IETF), em conjunto com o \textit{Internet Research Task Force} (IRTF), é o principal fórum internacional de padronização de tecnologias para a Internet. Apesar da importância do Brasil no contexto internacional (e especialmente na área de Ciência da Computação), a participação de brasileiros no IETF é pequena. Dificuldades financeiras podem ser citadas para tal participação, já que as oportunidades de financiamento são menores do que aquelas oferecidas em outros países. Neste contexto, a Chamada Pública 0001/2014\footnote{\textit{Seleção Pública de Propostas para Participação em Grupos de Trabalho e Reuniões do IETF/IRTF Janeiro/2014} - http://cgi.br/editais/ver/2}, financiada pelo CGI.br, proporciona um auxílio significativo para o desenvolvimento de atividades conjuntas entre o IETF/IRTF e membros de instituições de pesquisa e fabricantes de equipamentos baseados no Brasil.

%

A Chamada Pública 0001/2014 selecionou 4 participações nas reuniões do IETF: o projeto ``Utilização de Tecnologia Par-a-Par para Controle Autonômico Distribuídos de Sondas de Monitoramento em Redes de Computadores", dos autores Jéferson Nobre e Lisandro Granville, representando a Universidade Federal do Rio Grande do Sul (UFRGS); o projeto ``Metrologia e ciência de redes aplicadas à Internet", dos autores Klaus Wehmuth e Artur Ziviani, representando o Laboratório Nacional de Computação Científica (LNCC/MCTI); e o projeto ``Redes Definidas por Software considerando cenários com Virtualização de Funções da Rede", dos autores Marcelo Santos, Felipe Lopes, Leonidas Lima e Stênio Fernandes, representando a Universidade Federal de Pernambuco (UFPE). Além destes, também foi selecionado o apoio à participação de Denis Michellis, na modalidade de profissionais vinculados à indústria de redes nacional, funcionário da instituição Telefônica Brasil SA.

% organização

O presente relato está organizado da seguinte forma. Inicialmente, será realizada a descrição das atividades realizadas no contexto da Chamada Pública 0001/2014 do CGI.br. Tal descrição está dividida por questões didáticas de acordo com os diferentes projetos aprovados nessa chamada. Em seguida, serão apresentados comentários finais e algumas possibilidades a serem desenvolvidas.

\section{Descrição das Atividades Realizadas}

\subsection{Utilização de Tecnologia Par-a-Par para Controle Autonômico Distribuídos de Sondas de Monitoramento em Redes de Computadores}

% Início da relação

O suporte para requisitos de nível de serviço de rede tornou-se uma preocupação crítica em vários Grupos de Trabalho (\textit{Working Groups} - WGs) do IETF e grupos de pesquisa (\textit{Research Groups} - RGs)do IRTF. O tema da tese de doutorado de um dos autores, o qual era aluno da Universidade Federal do Rio Grande do Sul (UFRGS), está relacionada com o controle de tais requisitos (monitoramento de níveis de serviço) através de mecanismos de medição ativa de rede (e.g., \textit{One-Way Active Measurement Protocol} (OWAMP) \cite{OWAMP-Shalunov-2006}, \textit{Two-Way Active Measurement Protocol} (TWAMP) \cite{TWAMP-Hedayat-2008} e \textit{Cisco Service Level Assurance Protocol} (SLA) \cite{IPSLA-Chiba-2013}). Neste contexto, aspectos preliminares de tal tese foram apresentados em duas reuniões do Grupo de Pesquisa em Gerenciamento de Redes (\textit{Network Management Research Group} - NMRG) do IRTF: o NMRG 31 \textsuperscript{st} \textit{Meeting} (1\textsuperscript{st} \textit{Workshop on Large Scale Network Measurements}) \cite{P2PBNM-Nobre-2013a}, e a 32\textsuperscript{nd} NMRG \textit{Meeting} (\textit{Autonomics for Network Management}) \cite{P2PBNM-Nobre-2013b}, apresentações realizadas pelos autores. Essas apresentações confirmaram que havia interesse no contexto da IETF/IRTF aos aspectos de monitoramento de nível de serviço investigados na tese de doutorado citada. Além das apresentações, as interações com membros de diferentes WGs, como o IP \textit{Performance Metrics} (IPPM) WG e o Large-Scale Measurement of Broadband Performance (LMAP) WG, também demostraram que a participação nas reuniões do IETF poderia ser benéfica para a pesquisa realizada pelos autores. No entanto, o apoio financeiro necessário para tal participação é essencial, tal como o fornecido pela Chamada Pública do CGI.br 0001/2014.

% ANIMA/UCAN

A discussão sobre o tema \textit{Autonomics for Network Management}, proposta pelo NMRG, o \textit{Use Cases for Autonomic Networking} (UCAN) Bird of a Feather (BoF). Tal BOF foi destinado para discutir casos de uso de \textit{Autonomic Networking} (AN) com a comunidade do IETF assim como identificar outros possíveis casos. O objetivo fundamental da AN é o \textit{self-management}, incluindo propriedades auto-CHOP (\textit{self-configuration}, \textit{self-healing}, \textit{self-optimization}, e \textit{self-protection}), a fim de minimizar a dependência de administradores humanos e sistemas de gerenciamento centralizados. Um dos recursos de interesse apontados pelos coordenadores UCAN foi a capacidade de entidades distribuídas para apresentar processos de tomada de decisão adaptativos com base em informações e conhecimentos adquiridos a partir do seu ambiente, o que que está diretamente relacionado com a tese de um dos autores. Neste contexto, houve uma oportunidade para escrever um Internet-Draft (I-D), descrevendo a detecção distribuída de violações de SLA \cite{NMRG-Nobre-2015}.  Este I-D foi apresentado no UCAN BoF e foi adotado pela NMRG (atualmente em \textit{Working Group Last Call}). Posteriormente, tal BoF propiciou a  a formação do \textit{Autonomic Networking Integrated Model and Approach} (ANIMA) WG.

% Kostas

Diversos documentos foram produzido no NMRG para transformar conceitos abstratos de AN em requisitos concretos. No entanto, há uma discussão em curso neste RG sobre o conjunto inicial de definições empregadas em AN e como avançar para a padronização de aspectos de AN. Assim, um I-D foi proposto para revisitar a terminologia de AN estabelecida na literatura da comunidade de gerenciamento de redes \cite{NMRG-Pentikousis-2015}. Nesse contexto, um dos autores do presente trabalho foi convidado a contribuir para tal I-D, especialmente no que se refere aos aspectos de monitoramento autonômico. O I-D foi apresentado nos 35\textsuperscript{th} e 36\textsuperscript{th} NMRG \textit{meetings}, a fim de coletar feedback. O próximo passo para tal I-D é pedir a adoção pelo NMRG.

% futuro

Novas possibilidades de trabalhos no contexto do IETF são possíveis considerando a utilização de tecnologia par-a-par para controle autonômico distribuído de sondas de monitoramento em redes de computadores. Por exemplo, as iniciativas proposta pelo \textit{Large Scale Measurement of Broadband Performance} (LMAP) WG para padronizar mecanismos de medição de desempenho em banda larga podem ser dificultadas devido aos recursos humanos necessário. Dessa forma, um controle autonômico poderia auxiliar na operação em larga escala de tais mecanismos. Finalmente, pretende-se realizar a escrita de um I-D para o ANIMA WG descrevendo um \textit{Autonomic Service Agent} (ASA) que execute o controle autonômico distribuído de medições usando a infraestrutura ANIMA (e.g., o \textit{Autonomic Control Plane} - ACP).

\subsection{Metrologia e Ciência de Redes Aplicadas à Internet}

\subsection{Redes Definidas por Software Considerando Cenários com Virtualização de Funções da Rede}

\section{Atividades Conjuntas}

% TODO
% Participação nos encontros IETF-LAC

% TODO
% palestras, etc

% TODO
% remote hubs

A participação nos WGs e RGs facilita a organização de hubs remotos para os IETF \textit{meetings}. Tais hubs podem ampliar a audiência nesses \textit{meetings} e também a participação brasileira como um todo.

\section{Comentários Finais}

% TODO

A participação dos pesquisadores dos projetos contemplados on the IETF a Chamada Pública 0001/2014 do Comitê Gestor da Internet no Brasil foi benéfica para os próprios pesquisadores assim como para a inserção brasileira no IETF. A experiência ganha em tal participação tem sido compartilhada nas instituições dos pesquisadores, o que pode auxiliar no desenvolvimento e padronização de protocolos para a Internet.

\bibliographystyle{sbc}
\bibliography{phd-bibtex}

\end{document}

%%%
